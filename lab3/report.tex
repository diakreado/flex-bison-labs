\include{settings}

\begin{document}

\include{titlepage}

\section{Цель работы}
\begin{itemize}
\item Познакомиться с языком синтаксического разбора $yacc$
\item Выполнить трансляцию предложенных программ на языке $yacc$. Протестировать их работу.
\end{itemize}


\section{Ход работы}

\subsection{Проверка формата даты}

    Используемая программа:
    \lstinputlisting {src/_date/v1/v1.y}
    \lstinputlisting {src/_date/v1/v1.l}

    Сгенерированный файл:
    \lstinputlisting {src/_date/v1/zz.c}

    Входные данные:
    \lstinputlisting {src/_date/v1/test.in}

    Программа отработала корректно. 
    Так как ошибок обнаружено не было,
    то вывод программы тоже оказался пустым.

    Входные данные были модифицированы:
    \lstinputlisting {src/_date/v1/test1.in}

    В итоге, в консоль было выведено сообщение $?-syntax error$.
    Что сведетельствует, о несоответствии входных данных
    разработаному шаблону формата даты.

\subsection{Изменение структуры ввода даты}

    Программа приведенная в предыдущем задании была модифицирована:
    \lstinputlisting {src/_date/v2/v2.y}
    \lstinputlisting {src/_date/v2/v2.l}

    Входные данные, которые привели к успешному завершению:
    \lstinputlisting {src/_date/v2/test.in}

    Входные данные, которые привели к ошибке:
    \lstinputlisting {src/_date/v2/test1.in}


\subsection{Доступ к семантическим значениям}

    Используемая программа:
    \lstinputlisting {src/_date/v3/a/v3a.y}
    \lstinputlisting {src/_date/v3/a/v3.l}

    На вход программы было подано:
    \lstinputlisting {src/_date/v3/a/test1.in}

    В итоге получили:
    \lstinputlisting {src/_date/v3/a/test1.out}

    Следовательно можно сделать вывод, что программа отработала корректно.


\subsection{Проверка даты и вывод количества дней от 1970 г.}

    Используемая программа:
    \lstinputlisting {src/_date/v3/b/v3b.y}
    \lstinputlisting {src/_date/v3/b/v3.l}

    На вход программы было подано:
    \lstinputlisting {src/_date/v3/b/test.in}

    В итоге получили:
    \lstinputlisting {src/_date/v3/b/test.out}

    Следовательно можно сделать вывод, что программа отработала корректно.


\subsection{Семантическое значение date и вычисление разницы между датами}

    Используемая программа:
    \lstinputlisting {src/_date/v3/d/v3c.y}
    \lstinputlisting {src/_date/v3/d/v3.l}

    На вход программы было подано:
    \lstinputlisting {src/_date/v3/d/test.in}

    В итоге получили:
    \lstinputlisting {src/_date/v3/d/test.out}

    Следовательно можно сделать вывод, что программа отработала корректно.

\subsection{Определение сопутствующего типа значения нескольких типов}

    Используемая программа:
    \lstinputlisting {src/_date/v5/v5.y}
    \lstinputlisting {src/_date/v5/v5.l}

    На вход программы было подано:
    \lstinputlisting {src/_date/v5/test.in}

    В итоге получили:
    \lstinputlisting {src/_date/v5/test.out}

    Следовательно можно сделать вывод, что программа отработала корректно.


\subsection{Рекурсивные правила}

    Используемая программа:
    \lstinputlisting {src/list/v0/c1.y}
    \lstinputlisting {src/list/v0/c1.l}

    На вход программы было подано:
    \lstinputlisting {src/list/v0/test.in}

    В итоге получили:
    \lstinputlisting {src/list/v0/test.out}

    Следовательно можно сделать вывод, что программа отработала корректно.


\subsection{Использование рекурсии при чтении списка}

    Используемая программа:
    \lstinputlisting {src/list/v1/c1.y}
    \lstinputlisting {src/list/v1/c.l}

    На вход программы было подано:
    \lstinputlisting {src/list/v1/test.in}

    В итоге получили:
    \lstinputlisting {src/list/v1/test.out}

    Следовательно можно сделать вывод, что программа отработала корректно.


\subsection{Дополнительное микрозадание}


\section{Вывод}


\end{document}
