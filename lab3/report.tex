\documentclass[a4paper,14pt]{extarticle}

\usepackage[utf8]{inputenc}
\usepackage[russian]{babel}
\usepackage{graphicx}
\usepackage[top=0.8in, bottom=0.8in, left=0.8in, right=0.8in]{geometry}
\usepackage{pgfplots}
\usepackage{amsmath}
\usepackage{setspace}
\usepackage{titlesec}
\usepackage{float}
\usepackage{chngcntr}
\usepackage{pgfplots}
\usepackage{amsfonts}
\usepackage{pgfplotstable}
\usepackage{multirow}
\usepackage{karnaugh-map}
\usepackage{tikz,xcolor}
\usepackage{listings}
\usepackage{indentfirst}

\lstset{ %
extendedchars=\true,
keepspaces=true,
%language=no, % choose the language of the code
basicstyle=\footnotesize, % the size of the fonts that are used for the code
numbers=left, % where to put the line-numbers
numberstyle=\footnotesize, % the size of the fonts that are used for the line-numbers
stepnumber=1, % the step between two line-numbers. If it is 1 each line will be numbered
numbersep=10pt, % how far the line-numbers are from the code
backgroundcolor=\color{white}, % choose the background color. You must add \usepackage{color}
showspaces=false, % show spaces adding particular underscores
showstringspaces=false, % underline spaces within strings
showtabs=false, % show tabs within strings adding particular underscores
frame=single, % adds a frame around the code
tabsize=4, % sets default tabsize to 2 spaces
captionpos=b, % sets the caption-position to bottom
breaklines=true, % sets automatic line breaking
breakatwhitespace=false, % sets if automatic breaks should only happen at whitespace
escapeinside={\%*}{*)}, % if you want to add a comment within your code
%postbreak=\raisebox{0ex}[0ex][0ex]{\ensuremath{\color{red}\hookrightarrow\space}}
}

\titleformat{\section}[hang]
  {\bfseries}
  {}
  {0em}
  {\hspace{-0.4pt}\large \thesection\hspace{0.6em}}
  
  
\titleformat{\subsection}[hang]
  {\bfseries}
  {}
  {0em}
  {\hspace{-0.4pt}\large \thesubsection\hspace{0.6em}}

%\linespread{1.3} % полуторный интервал
%\renewcommand{\rmdefault}{ftm} % Times New Roman

\newcommand{\nx}{\overline{x}}
\newcommand{\p}{0.31}
\newcommand{\scale}{1.4}

\counterwithin{figure}{section}
\counterwithin{equation}{section}
\counterwithin{table}{section}



\begin{document}


\begin{titlepage}
\centering
Санкт-Петербургский политехнический университет Петра Великого \\
\vspace{0.15cm}
Кафедра компьютерных систем и программных технологий \\
\vspace{6.5cm}

{\centering \textbf{Отчёт по лабораторной работе} \\ 
\vspace{0.15cm}
\textbf{Дисциплина}: Транслирующие системы \\
\vspace{0.15cm}
\textbf{Тема}: Транслятор операторов $while$ языка $C$ } \\

\vspace{6.5cm}

\begin{table}[H]
\begin{tabular}{p{\textwidth}@{}r}
{Выполнил студент гр. 43501/3} \hfill {Мальцев  М.С.} \\
{Преподаватель} \hfill {Цыган В.Н.} \\
\end{tabular}
\end{table}
\vfill

{\centering Санкт-Петербург \\ 
\vspace{0.15cm}
\today}
\end{titlepage}

\section{Цель работы}
\begin{itemize}
\item Познакомиться с языком синтаксического разбора $yacc$
\item Выполнить трансляцию предложенных программ на языке $yacc$. Протестировать их работу.
\end{itemize}


\section{Ход работы}

\subsection{Проверка формата даты}

    Используемая программа:
    \lstinputlisting {src/_date/v1/v1.y}
    \lstinputlisting {src/_date/v1/v1.l}

    Сгенерированный файл:
    \lstinputlisting {src/_date/v1/zz.c}

    Входные данные:
    \lstinputlisting {src/_date/v1/test.in}

    Программа отработала корректно. 
    Так как ошибок обнаружено не было,
    то вывод программы тоже оказался пустым.

    Входные данные были модифицированы:
    \lstinputlisting {src/_date/v1/test1.in}

    В итоге, в консоль было выведено сообщение $?-syntax error$.
    Что сведетельствует, о несоответствии входных данных
    разработаному шаблону формата даты.

\subsection{Изменение структуры ввода даты}

    Программа приведенная в предыдущем задании была модифицирована:
    \lstinputlisting {src/_date/v2/v2.y}
    \lstinputlisting {src/_date/v2/v2.l}

    Входные данные, которые привели к успешному завершению:
    \lstinputlisting {src/_date/v2/test.in}

    Входные данные, которые привели к ошибке:
    \lstinputlisting {src/_date/v2/test1.in}


\subsection{Доступ к семантическим значениям}

    Используемая программа:
    \lstinputlisting {src/_date/v3/a/v3a.y}
    \lstinputlisting {src/_date/v3/a/v3.l}

    На вход программы было подано:
    \lstinputlisting {src/_date/v3/a/test1.in}

    В итоге получили:
    \lstinputlisting {src/_date/v3/a/test1.out}

    Следовательно можно сделать вывод, что программа отработала корректно.


\subsection{Проверка даты и вывод количества дней от 1970 г.}

    Используемая программа:
    \lstinputlisting {src/_date/v3/b/v3b.y}
    \lstinputlisting {src/_date/v3/b/v3.l}

    На вход программы было подано:
    \lstinputlisting {src/_date/v3/b/test.in}

    В итоге получили:
    \lstinputlisting {src/_date/v3/b/test.out}

    Следовательно можно сделать вывод, что программа отработала корректно.


\subsection{Семантическое значение date и вычисление разницы между датами}

    Используемая программа:
    \lstinputlisting {src/_date/v3/d/v3c.y}
    \lstinputlisting {src/_date/v3/d/v3.l}

    На вход программы было подано:
    \lstinputlisting {src/_date/v3/d/test.in}

    В итоге получили:
    \lstinputlisting {src/_date/v3/d/test.out}

    Следовательно можно сделать вывод, что программа отработала корректно.

\subsection{Определение сопутствующего типа значения нескольких типов}

    Используемая программа:
    \lstinputlisting {src/_date/v5/v5.y}
    \lstinputlisting {src/_date/v5/v5.l}

    На вход программы было подано:
    \lstinputlisting {src/_date/v5/test.in}

    В итоге получили:
    \lstinputlisting {src/_date/v5/test.out}

    Следовательно можно сделать вывод, что программа отработала корректно.


\subsection{Рекурсивные правила}

    Используемая программа:
    \lstinputlisting {src/list/v0/c1.y}
    \lstinputlisting {src/list/v0/c1.l}

    На вход программы было подано:
    \lstinputlisting {src/list/v0/test.in}

    В итоге получили:
    \lstinputlisting {src/list/v0/test.out}

    Следовательно можно сделать вывод, что программа отработала корректно.


\subsection{Использование рекурсии при чтении списка}

    Используемая программа:
    \lstinputlisting {src/list/v1/c1.y}
    \lstinputlisting {src/list/v1/c.l}

    На вход программы было подано:
    \lstinputlisting {src/list/v1/test.in}

    В итоге получили:
    \lstinputlisting {src/list/v1/test.out}

    Следовательно можно сделать вывод, что программа отработала корректно.


\subsection{Дополнительное микрозадание}


\section{Вывод}


\end{document}
