\documentclass[a4paper,14pt]{extarticle}

\usepackage[utf8]{inputenc}
\usepackage[russian]{babel}
\usepackage{graphicx}
\usepackage[top=0.8in, bottom=0.8in, left=0.8in, right=0.8in]{geometry}
\usepackage{pgfplots}
\usepackage{amsmath}
\usepackage{setspace}
\usepackage{titlesec}
\usepackage{float}
\usepackage{chngcntr}
\usepackage{pgfplots}
\usepackage{amsfonts}
\usepackage{pgfplotstable}
\usepackage{multirow}
\usepackage{karnaugh-map}
\usepackage{tikz,xcolor}
\usepackage{listings}
\usepackage{indentfirst}

\lstset{ %
extendedchars=\true,
keepspaces=true,
%language=no, % choose the language of the code
basicstyle=\footnotesize, % the size of the fonts that are used for the code
numbers=left, % where to put the line-numbers
numberstyle=\footnotesize, % the size of the fonts that are used for the line-numbers
stepnumber=1, % the step between two line-numbers. If it is 1 each line will be numbered
numbersep=10pt, % how far the line-numbers are from the code
backgroundcolor=\color{white}, % choose the background color. You must add \usepackage{color}
showspaces=false, % show spaces adding particular underscores
showstringspaces=false, % underline spaces within strings
showtabs=false, % show tabs within strings adding particular underscores
frame=single, % adds a frame around the code
tabsize=4, % sets default tabsize to 2 spaces
captionpos=b, % sets the caption-position to bottom
breaklines=true, % sets automatic line breaking
breakatwhitespace=false, % sets if automatic breaks should only happen at whitespace
escapeinside={\%*}{*)}, % if you want to add a comment within your code
%postbreak=\raisebox{0ex}[0ex][0ex]{\ensuremath{\color{red}\hookrightarrow\space}}
}

\titleformat{\section}[hang]
  {\bfseries}
  {}
  {0em}
  {\hspace{-0.4pt}\large \thesection\hspace{0.6em}}
  
  
\titleformat{\subsection}[hang]
  {\bfseries}
  {}
  {0em}
  {\hspace{-0.4pt}\large \thesubsection\hspace{0.6em}}

%\linespread{1.3} % полуторный интервал
%\renewcommand{\rmdefault}{ftm} % Times New Roman

\newcommand{\nx}{\overline{x}}
\newcommand{\p}{0.31}
\newcommand{\scale}{1.4}

\counterwithin{figure}{section}
\counterwithin{equation}{section}
\counterwithin{table}{section}



\begin{document}


\begin{titlepage}
\centering
Санкт-Петербургский политехнический университет Петра Великого \\
\vspace{0.15cm}
Кафедра компьютерных систем и программных технологий \\
\vspace{6.5cm}

{\centering \textbf{Отчёт по лабораторной работе} \\ 
\vspace{0.15cm}
\textbf{Дисциплина}: Транслирующие системы \\
\vspace{0.15cm}
\textbf{Тема}: Транслятор операторов $while$ языка $C$ } \\

\vspace{6.5cm}

\begin{table}[H]
\begin{tabular}{p{\textwidth}@{}r}
{Выполнил студент гр. 43501/3} \hfill {Мальцев  М.С.} \\
{Преподаватель} \hfill {Цыган В.Н.} \\
\end{tabular}
\end{table}
\vfill

{\centering Санкт-Петербург \\ 
\vspace{0.15cm}
\today}
\end{titlepage}

\section{Цель работы}
\begin{itemize}
\item Познакомиться с генератором программ лексической обработки текстов Lex.
\item Выполнить трансляцию предложенных программ на языке Lex. Проте-стировать их работу.
\end{itemize}


\section{Ход работы}
\subsection{Удаление пробелов и табуляций в начале строк}
Рассмотрим программу, которая передает в выходной поток все литеры входного потока кроме пробелов и табуляций в начале строки.
\lstinputlisting {src/ex01.l}

Тестовые данные, подаваемые на вход, приведены ниже:
\lstinputlisting {src/ex01.in}

В результате получили:
\lstinputlisting {src/ex01.out}

Программа отработала корректно. Пробелы и символы табуляции были заменены.

\subsection{Подсчет числа строк}
Программа подсчитывающая количество строк:
\lstinputlisting {src/ex02.l}

Тестовые данные, подаваемые на вход, приведены ниже:
\lstinputlisting {src/ex02.in}

В результате получили:
\lstinputlisting {src/ex02.out}

Программа отработала корректно. Количество строк было подсчитано верно.

\subsection{Подсчет и вывод знаковых целых чисел}
Программа подсчитывающая и выводящая количество знаковых чисел:
\lstinputlisting {src/ex03.l}

Тестовые данные, подаваемые на вход, приведены ниже:
\lstinputlisting {src/ex03.in}

В результате получили:
\lstinputlisting {src/ex03.out}

Программа отработала корректно. Видно, что при обнаружении целого числа программа добавляет к нему индекс.

\subsection{Вывод идентификаторов и беззнаковых целых чисел}
Используемая программа:
\lstinputlisting {src/ex04.l}

Тестовые данные, подаваемые на вход, приведены ниже:
\lstinputlisting {src/ex04.in}

В результате получили:
\lstinputlisting {src/ex04.out}

Программа отработала корректно. Все идентификаторы и беззнаковые целые числа были отдельно выведены.

\subsection{Подсчет и вывод гистограммы длин слов}
Текст используемой программы:
\lstinputlisting {src/ex05.l}

Тестовые данные, подаваемые на вход, приведены ниже:
\lstinputlisting {src/ex05.in}

В результате получили:
\lstinputlisting {src/ex05.out}

Программа отработала корректно. Была выведенна длинна слов и их количество.

\subsection{Дополнительное задание}
Цепочки из 0 и 1 не больше, чем 2 единицы подряд.

Разработанная программа:
\lstinputlisting {src/ex05_1.l}

Тестовые данные, подаваемые на вход, приведены ниже:
\lstinputlisting {src/ex05_1.in}

В результате получили:
\lstinputlisting {src/ex05_1.out}

Программа отработала корректно.

\subsection{Вывод строки наискосок при помощи yyless}
Текст используемой программы:
\lstinputlisting {src/ex06.l}

\begin{itemize}
    \item Тест 1
    
    Тестовые данные, подаваемые на вход, приведены ниже:
    \lstinputlisting {src/ex06.in}
    
    В результате получили:
    \lstinputlisting {src/ex06.out}

    \item Тест 2
    
    Тестовые данные, подаваемые на вход, приведены ниже:
    \lstinputlisting {src/ex06_1.in}
    
    В результате получили:
    \lstinputlisting {src/ex06_1.out}
\end{itemize}

Программа отработала корректно.

\subsection{Макросы и ввод-вывод низкого уровня}
    Программа, приведенная ниже, идентифицирует буквы и числа в 10-м и 16-м фарматах. Программа игнорирует комментарии.
    \lstinputlisting {src/ex07_1.l}
    
    Тестовые данные, подаваемые на вход, приведены ниже:
    \lstinputlisting {src/ex07_1.in}
    
    В результате получили:
    \lstinputlisting {src/ex07_1.out}

    Программа отработала корректно.

\subsection{Проверка конца входного потока при использовании input}
    Настоящая программа, является дополнением предыдущей. 
    В качестве нововведения добавлена проверка конца входного потока при использо-вании input.
    Проблема предшествующей программа заключалась в том, что происходило зацикливание при незакрытом комментарии.
    Код модифицированной программы представлен ниже:
    \lstinputlisting {src/ex07_2.l}
    
    Тестовые данные, подаваемые на вход, приведены ниже:
    \lstinputlisting {src/ex07_2.in}
    
    В результате получили:
    \lstinputlisting {src/ex07_2.out}

    Программа отработала корректно.

\subsection{Функция unput}
    Приведенная ниже программа производит реверсирование идентификаторов, начинающихся с символа @:
    \lstinputlisting {src/ex08_2.l}

    Тестовые данные, подаваемые на вход, приведены ниже:
    \lstinputlisting {src/ex08_1.in}
    
    В результате получили:
    \lstinputlisting {src/ex08_1.out}

    Программа отработала корректно.

\subsection{Двусмысленный набор правил}
    Приведенная ниже программа демонстрирует выбор правила при двусмысленном наборе.
    \lstinputlisting {src/ex09.l}

    Тестовые данные, подаваемые на вход, приведены ниже:
    \lstinputlisting {src/ex09.in}
    
    В результате получили:
    \lstinputlisting {src/ex09.out}

    Программа отработала корректно.

\subsection{Неправильный шаблон для распознавания строки в кавычках}
    Ниже приведена программа, которая демонстрирует неверное определение
    шаблона, в задаче выявления строк, не заключенных в кавычки.
    \lstinputlisting {src/ex10.l}

    Тестовые данные, подаваемые на вход, приведены ниже:
    \lstinputlisting {src/ex10.in}
    
    В результате получили:
    \lstinputlisting {src/ex10.out}

    Программа отработала некорректно, так как вместо того, чтобы исключить слова 
    выделенные в кавычках она исключила и всё, что в пределах строки между выделенными 
    словами находилось.

\subsection{Правильный шаблон для распознавания строки в кавычках}
    
    Приведенная ниже программа демонстрирует работу правильного шаблона для распознования строки в кавычках.
    \lstinputlisting {src/ex11.l}

    Тестовые данные, подаваемые на вход, приведены ниже:
    \lstinputlisting {src/ex10.in}
    
    В результате получили:
    \lstinputlisting {src/ex11.out}

    Программа отработала корректно.

\subsection{Использование переменной состояния}

    Приведенная ниже программа заменяет слово magic на first, second, third,
    в зависимости от того, какая цифра стоит в начале строки.
    \lstinputlisting {src/ex12.l}

    Тестовые данные, подаваемые на вход, приведены ниже:
    \lstinputlisting {src/ex12.in}
    
    В результате получили:
    \lstinputlisting {src/ex12.out}

    Программа отработала корректно.

\subsection{Решение предыдущей задачи при помощи стартовых условий}

    Программа была модифицирована для решения той же самой задачи,
    но уже с использованием стартовых условий.
    \lstinputlisting {src/ex13_1.l}

    Тестовые данные, подаваемые на вход, приведены ниже:
    \lstinputlisting {src/ex12.in}
    
    В результате получили:
    \lstinputlisting {src/ex13_1.out}

    Программа отработала корректно. Результат был получен тот же самый.

\subsection{Трассировка стартовых условий}

    Опять решается таже задача с модифицированной программой. Добавлена трассировка стартовых условий.
    \lstinputlisting {src/ex13_2.l}

    Тестовые данные, подаваемые на вход, приведены ниже:
    \lstinputlisting {src/ex12.in}
    
    В результате получили:
    \lstinputlisting {src/ex13_2.out}

    Программа отработала корректно. Результат был получен тот же самый.
    
    Кроме того, программа вывела в консоль следующее сообщение:

    <0><0><0><0><0><0><0><0><0><0><0><0>

    <0><0><0><0><0><0><0><0><0><0><0><0>

    <0><0><0><0><0><0><0><0><0><0><0><0>

    <0><0><0><0><0><0><0><0><0><0><0><0>

    <0><0><0><0><0><0><0><0><0><0><0><0>

    <0><0><0><0><0><0><1><1><1><1><1><1>

    <1><1><1><1><1><1><1><1><1><1><1><1>

    <1><1><1><1><1><1><1><1><1><1><1><1>

    <1><1><1><1><1><1><1><1><1><1><1><1>

    <1><1><1><1><0><3><3><3><3><3><3><3>

    <3><3><3><3><3><3><3><3><3><3><3><3>

    <3><3><3><3><3><3><3><3><3><3><3><3>

    <3><3><3><3><3><3><3><3><3><3><3><3>

    <3><3><3><0><2><2><2><2><2><2><2><2>

    <2><2><2><2><2><2><2><2><2><2><2><2>

    <2><2><2><2><2><2><2><0><0><0><0><0>

    <0><0><0><0><2><2><2><2><2><2><2><2>

    <2><2><2><2><2><2><2><2>
    
\subsection{Подсчет количества she и he без учета he внутри she}

    Приведенная ниже программа вычисляет количество he и she в тексте.
    \lstinputlisting {src/ex14_1.l}

    Тестовые данные, подаваемые на вход, приведены ниже:
    \lstinputlisting {src/ex14_1.in}
    
    В результате получили:
    \lstinputlisting {src/ex14_1.out}

    Программа отработала корректно.

\subsection{Подсчет всех экземпляров she и he с использованием REJECT}

    Приведенная выше программа была модифицированна с использованием команды REJECT.
    \lstinputlisting {src/ex14_2.l}

    Тестовые данные, подаваемые на вход, приведены ниже:
    \lstinputlisting {src/ex14_1.in}
    
    В результате получили:
    \lstinputlisting {src/ex14_2.out}

    Программа отработала корректно. Результат изменился относительно предыдущего пункта. Это связано с тем, что теперь в слове $she$ мы находим $he$.

\subsection{Подсчет she и he с использованием yyless}

    Приведенная выше программа была модифицированна с заменой REJECT на yyless.
    \lstinputlisting {src/ex14_3.l}

    Тестовые данные, подаваемые на вход, приведены ниже:
    \lstinputlisting {src/ex14_1.in}
    
    В результате получили:
    \lstinputlisting {src/ex14_3.out}

    Программа отработала корректно. Результат соответствует тому, что мы получили ранее.


\section{Вывод}

В результате выполнения работы были получены навыки работы с генератором программ лексической обработки текстов Lex. 
Было изучено и запущено четырнадцать программ, тексты которых были распространенны преподавателем. Также была разработана собственная программа,
в соответствии с индивидуальным заданием. 
Программы были транслированы в текста на языке С, скомпилированы и запущены с различными входными данными. Эксперименты, которые наиболее полно
отражают суть работы программы, приведены в отчёте.
Проведенная работа позволила изучить проблемы, возникающие при разработке и шаблоны,
которые могу служить решением для них.

\end{document}
