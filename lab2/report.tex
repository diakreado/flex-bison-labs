\include{settings}

\begin{document}

\include{titlepage}

\section{Задание}
    Составить LEX-программу для следующего перевода.
    Пусть для записи химических формул используются следующие восемь элементов
    
    \textbf{H, C, N, O, SI, S, CL, SN.}

    Элементы в формулах разделяются запятыми.
    Элементы могут появляться в любом порядке и в любых сочетаниях.
    Для указания количества атомов в формуле используется цифр,
    записанная вслед за обозначением химического элемента.
    Формулы не обязательно представляют реально существующие соединения. 
    Несколько примеров записи формул:

    \textbf{H2, O}

    \textbf{O, H7}

    \textbf{SN, S, O4}

    Таким образом имеется девять входных символов:

    \textbf{C H I L N O S, $\mu$}

    где $\mu$ - обозначение цифры.

    Необходимо осуществлять распознование допустимых 
    (в рамках описанных выше правил) химических формул и
    вычислять молекулярный вес вещества, описанной формулой.
    
    Входные предложения, которые не являются допустимыми должны печататься
    с соответствущим диагностическими сообщениями.

    Предусмотреть не менее двух разных диагностических сообщений.


\section{Ход работы}

\subsection{Применяемый шаблон}

    Для определения указаных химических элементов, был разработан шаблон:

    $H[2-9]?("{},\ "{}|"\backslash n")$

    Вместо $H$ подставляются остальные химические элементы.

    По условию, за буквенным именованием элемента может идти его номер.
    Номер - это цифра, но так как в данном случае у нас не может идти ни
    цифра 1(т.к. она в таком случае она не несет никакой информации),
    ни цифра 0(т.к. это означает отсутствие элемента, что в рамках логики
    приложения бессмыслено), то диапазон выбран от 2 до 9.
    
    Заканчиваться конструкция может двумя способами: либо запятой и пробелом,
    что предполагает следующий элемент в формуле, либо символом окончания строки,
    что говорит об окончании формулы.

\subsection{Разработанные функции}

    Для более удобной обработки текста была разработана функция:

    $void\ handler(int ,\ int massOfOne)$

    Её задача определение молекулярный массы элемента и формирование
    значения молекулярной массы формулы.
    Также в задачи решаемые функцией входит возвращения последнего элемента
    в поток обработки, в случае, если этот элемент - это символ перевода строки.
    Это сделано для того, чтобы обработчик конца формулы смог зафиксировать её
    и вывести результат вычисления молекулярной массы либо диагностическое сообщение.

    Первый параметр $lenOfElement$ обозначает длину элемента в символах.
    Например, для элемента $H$ он будет равен 1, а для $CL$ - 2.

    Второй параметр $massOfOne$ обозначает молекулярную массу элемента.
    Для элемента $H$ он будет равен 1, а для $CL$ - 35.
    

\subsection{Вывод результата и диагностическое сообщение}

    Возможно два типа сообщений: сообщение о том, что формула обработалась корректно,
    содержащее значение молекулярной массы для формулы, либо сообщение о некорректности
    введенной формулы. В первом случае, вывод выглядит следующим образом:

    $H\ |\ 1\ massм$

    Во втором, помимо сообщения об ошибке, справа от некорректного символа выводится
    восклицательный знак и выводится краткое сообщение, почему она произошла.
    Программа может сообщить о трёх ошибках:

    $H5!H\ |\ Error\ -\ correct\ syntax,\ but\ incorrect\ lexis\ !!!$

    $l!o!l!\ !CL2\ |\ Error\ -\ incorrect\ syntax\ !!!$

    $O0!\ |\ Error\ -\ incorrect\ number\ !!!$

\subsection{Текст программы}

    \lstinputlisting {src/indiv.l}

\subsection{Тестирование программы}

    Тестовые входные данные были сформированы таким образом, чтобы максимально полно
    продемонстрировать функциональность программы.

    Входные данные:

    \lstinputlisting {src/indiv.in}

    Результат работы программы:

    \lstinputlisting {src/indiv.out}

    Полученные результаты соответствуют требованиям, указаным в задании.
    В связи с этим, можно считать, что программа успешно прошла тест.

\section{Вывод}

    В ходе работы была написана программа, которая проверяет корректность
    введённых химических формул, а в случае корректности вычисляет молекулярную массу формулы.
    Программа успешно справилась с тестовой задачей.

    В результате работы закреплены навыки работы с программой для генерации лексических анализаторов $Lex$.
    Получен опыт в разработке и отладке приложений основывающихся на этой программе.

\end{document}
