\documentclass[a4paper,14pt]{extarticle}

\usepackage[utf8]{inputenc}
\usepackage[russian]{babel}
\usepackage{graphicx}
\usepackage[top=0.8in, bottom=0.8in, left=0.8in, right=0.8in]{geometry}
\usepackage{pgfplots}
\usepackage{amsmath}
\usepackage{setspace}
\usepackage{titlesec}
\usepackage{float}
\usepackage{chngcntr}
\usepackage{pgfplots}
\usepackage{amsfonts}
\usepackage{pgfplotstable}
\usepackage{multirow}
\usepackage{karnaugh-map}
\usepackage{tikz,xcolor}
\usepackage{listings}
\usepackage{indentfirst}

\lstset{ %
extendedchars=\true,
keepspaces=true,
%language=no, % choose the language of the code
basicstyle=\footnotesize, % the size of the fonts that are used for the code
numbers=left, % where to put the line-numbers
numberstyle=\footnotesize, % the size of the fonts that are used for the line-numbers
stepnumber=1, % the step between two line-numbers. If it is 1 each line will be numbered
numbersep=10pt, % how far the line-numbers are from the code
backgroundcolor=\color{white}, % choose the background color. You must add \usepackage{color}
showspaces=false, % show spaces adding particular underscores
showstringspaces=false, % underline spaces within strings
showtabs=false, % show tabs within strings adding particular underscores
frame=single, % adds a frame around the code
tabsize=4, % sets default tabsize to 2 spaces
captionpos=b, % sets the caption-position to bottom
breaklines=true, % sets automatic line breaking
breakatwhitespace=false, % sets if automatic breaks should only happen at whitespace
escapeinside={\%*}{*)}, % if you want to add a comment within your code
%postbreak=\raisebox{0ex}[0ex][0ex]{\ensuremath{\color{red}\hookrightarrow\space}}
}

\titleformat{\section}[hang]
  {\bfseries}
  {}
  {0em}
  {\hspace{-0.4pt}\large \thesection\hspace{0.6em}}
  
  
\titleformat{\subsection}[hang]
  {\bfseries}
  {}
  {0em}
  {\hspace{-0.4pt}\large \thesubsection\hspace{0.6em}}

%\linespread{1.3} % полуторный интервал
%\renewcommand{\rmdefault}{ftm} % Times New Roman

\newcommand{\nx}{\overline{x}}
\newcommand{\p}{0.31}
\newcommand{\scale}{1.4}

\counterwithin{figure}{section}
\counterwithin{equation}{section}
\counterwithin{table}{section}



\begin{document}


\begin{titlepage}
\centering
Санкт-Петербургский политехнический университет Петра Великого \\
\vspace{0.15cm}
Кафедра компьютерных систем и программных технологий \\
\vspace{6.5cm}

{\centering \textbf{Отчёт по лабораторной работе} \\ 
\vspace{0.15cm}
\textbf{Дисциплина}: Транслирующие системы \\
\vspace{0.15cm}
\textbf{Тема}: Транслятор операторов $while$ языка $C$ } \\

\vspace{6.5cm}

\begin{table}[H]
\begin{tabular}{p{\textwidth}@{}r}
{Выполнил студент гр. 43501/3} \hfill {Мальцев  М.С.} \\
{Преподаватель} \hfill {Цыган В.Н.} \\
\end{tabular}
\end{table}
\vfill

{\centering Санкт-Петербург \\ 
\vspace{0.15cm}
\today}
\end{titlepage}

\section{Задание}
    Составить LEX-программу для следующего перевода.
    Пусть для записи химических формул используются следующие восемь элементов
    
    \textbf{H, C, N, O, SI, S, CL, SN.}

    Элементы в формулах разделяются запятыми.
    Элементы могут появляться в любом порядке и в любых сочетаниях.
    Для указания количества атомов в формуле используется цифр,
    записанная вслед за обозначением химического элемента.
    Формулы не обязательно представляют реально существующие соединения. 
    Несколько примеров записи формул:

    \textbf{H2, O}

    \textbf{O, H7}

    \textbf{SN, S, O4}

    Таким образом имеется девять входных символов:

    \textbf{C H I L N O S, $\mu$}

    где $\mu$ - обозначение цифры.

    Необходимо осуществлять распознование допустимых 
    (в рамках описанных выше правил) химических формул и
    вычислять молекулярный вес вещества, описанной формулой.
    
    Входные предложения, которые не являются допустимыми должны печататься
    с соответствущим диагностическими сообщениями.

    Предусмотреть не менее двух разных диагностических сообщений.


\section{Ход работы}

    Ну, пусть идёт

\end{document}
