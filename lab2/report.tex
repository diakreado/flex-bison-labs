\include{settings}

\begin{document}

\include{titlepage}

\section{Задание}
    Составить LEX-программу для следующего перевода.
    Пусть для записи химических формул используются следующие восемь элементов
    
    \textbf{H, C, N, O, SI, S, CL, SN.}

    Элементы в формулах разделяются запятыми.
    Элементы могут появляться в любом порядке и в любых сочетаниях.
    Для указания количества атомов в формуле используется цифр,
    записанная вслед за обозначением химического элемента.
    Формулы не обязательно представляют реально существующие соединения. 
    Несколько примеров записи формул:

    \textbf{H2, O}

    \textbf{O, H7}

    \textbf{SN, S, O4}

    Таким образом имеется девять входных символов:

    \textbf{C H I L N O S, $\mu$}

    где $\mu$ - обозначение цифры.

    Необходимо осуществлять распознование допустимых 
    (в рамках описанных выше правил) химических формул и
    вычислять молекулярный вес вещества, описанной формулой.
    
    Входные предложения, которые не являются допустимыми должны печататься
    с соответствущим диагностическими сообщениями.

    Предусмотреть не менее двух разных диагностических сообщений.


\section{Ход работы}

    Ну, пусть идёт

\end{document}
