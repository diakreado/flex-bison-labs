\include{settings}

\begin{document}

\include{titlepage}

\section{Задание}
    Транслятор операторов $while$ языка $C$:
    
    Тип данных: int.
    
    Условное выражение: арифметическое бесскобочнрое выражение,
    т.е. операции выполняются слева на право.

    В теле цикла:
    \begin{enumerate}
        \item операторы присваивания вида $id=$ бесскобочное арифметическое выражение
        \item вложенный оператор $while$
    \end{enumerate}

    Выходной продукт:
    \begin{enumerate}
        \item текст на языке ассембера A86
        \item тетрады матрицы синтаксического дерева
    \end{enumerate}

\section{Ход работы}

    Было решено использовать язык ассемблера для архитектуры x86. 
    В связи с тем, что эта архитектура наиболее распространена 
    и был найден удобный транслятор из языка С в язык ассембера x86.
    (\href{https://gcc.godbolt.org/}{https://gcc.godbolt.org/})


    Учитывая перечисленные требования была разработана программа:

    \lstinputlisting {src/c.l}
    \lstinputlisting {src/c1.y}
    \lstinputlisting {src/zz.c}
    \lstinputlisting {src/vector.c}
    \lstinputlisting {src/vector.h}

    В качестве входных данных было подано следующее:
    \lstinputlisting {src/test1.in}

    В результате работы программы получили следующее:
    \lstinputlisting {src/test1.out}

    Что соответствует результатам трансляции с помощью сторонней программы:
    \begin{figure}[h]
        \center{\includegraphics[scale=0.6]{img/3.png}}
        \label{fig:image}
    \end{figure}

    Следовательно, можно считать, что разработаная программа работает верно.

    Было предусмотренно два диагностических сообщения:

    \begin{enumerate}
        \item Сообщение о повторном объявлении переменной.
        
        В качестве входных данных было подано следующее:
        \lstinputlisting {src/test2.in}
    
        В результате работы программы получили следующее:
        \lstinputlisting {src/test2.out}

        \item Сообщение о исользовании не объявленной переменной.
        
        В качестве входных данных было подано следующее:
        \lstinputlisting {src/test4.in}
    
        В результате работы программы получили следующее:
        \lstinputlisting {src/test4.out}
    \end{enumerate}

\section{Вывод}
    В результате выполнения лабораторной работы был написан простейший 
    транслятор операторов $while$ языка $C$. Для написания программы
    использовались генераторы синтаксического и лексического разбора
    $yacc$ и $lex$. 
    В ходе выполнения работы сначала был выбран язык ассембера, в который
    будут транслироваться операторы языка С. 
    Далее был разработана часть отвечающая за лексический разбор, в след за
    ней была написана часть, в задачи которой входит синтаксический разбор.
    Получившаяся программа была протестирована на различных входных данных.
    Было создано два диагностических сообщения.
    Полученные результаты соответствуют тому, что было описано в задании.

    Проведенная работа позволила лучше понять принципы совместного
    использования генерторов для синтаксического и лексического разбора и
    принципы посторения трансляторов для языков программирования.
    Также в ходе выполнения лабораторной были получены навыки разработки
    приложений на основе этих языков.


\section{Используемая литература}
    \begin{itemize}
        \item John Levine. Flex \& Bison: Text Processing Tools. — O\'{}Reilly Media, 2009
        \item Программирование лексического и синтаксического разбора на языках C, Lex и Yacc — А.В. Жуков, 2014
    \end{itemize}

\end{document}
